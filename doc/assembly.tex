\documentclass[10pt, journal]{IEEEtran}
\title{Assembly of the uFluidic Magnetic Cell Counting Device}
\author{Michael Nolan, Fang-Chen Lin, Danyal Haider}

\begin{document}
\maketitle \par
\subsection{Description}
The goal of this procedure is to form a channel in a block of PDMS
with a coil of wire wrapped tightly around it.
\subsection{Materials}
\begin{itemize}
\item 3D printed mold
\item Hot plate
\item Scale or Balance
\item Scalpel
\item Tweezers
\item Dissecting Microscope
\item Hand Mixer
\item Needle Nose Pliers

\item Dow Sylgard 184 Silicone Encapsulant (PDMS)
\item Optical fibre
\item 44 or 46 AWG enameled copper wire
\item 50mL Centrifuge Tube
\item Electrical Tape
  
\end{itemize}

\subsection{Preparation of the optical fibre}
\begin{enumerate}
\item Suspend the optical fibre horizonatally under the viewing area
  of the microscope. The optical fibre should be positioned such
  that there is room above and below it to wrap the copper wire.

\item Cut a piece of copper wire about 40cm long, and tape one end
  directly below the optical fibre.

\item Using the taped end to provide a small amount of tension, begin
  wrapping the wire around the optical fibre.

\item Keep wrapping the wire until around 20 turns are on the optical
  fibre. Tweezers may help to keep the coils neat and bunched tightly.

\item Thread the optical fibre into the mold, by first threading one
  end, then the other into the holes from the inside of the mold to
  the outside.

\item Place the mold onto a piece of plastic, foil, or a petri dish to
  catch any PDMS spills.
      
\end{enumerate}

\subsection{Preparation of the PDMS}
Notes: PDMS is oily and sticky. Do this on a disposable surface such
as newspaper and wear gloves.

\begin{enumerate}
\item Prepare a 50mL centrifuge tube so it can be held upright on the scale or balance.
\item Weigh out around 10g of PDMS compound 1 (The larger one) into the tube. The exact amount isn't important, but the ratio between the two parts is.
\item Weigh out 1/10 of the amount weighed out in step 2 of the PDMS
  crosslinker (the small bottle) into the test tube.
\item Mix thoroughly using a stir stick. Don't worry about creating bubbles as they will be removed shortly.
\end{enumerate}

\subsection{De-gassing the PDMS (Vacuum Method)}
\begin{enumerate}
  \item Pour the PDMS mixture into the mold. Make sure to cover the
    coil and optical fibre.
  \item Place the mold into a vacuum chamber, making sure to place
    something under the mold to catch any leaks.
  \item Pull a vacuum on the chamber for 15-30 minutes to remove any bubbles

\end{enumerate}

\subsection{De-gassing the PDMS (Cheap Method)}
This method does not work as well as the vacuum pump method as it does
not remove any dissolved gasses in the PDMS. It does however do a
remarkable job of removing the bubbles introduced by mixing. It also
works best with a tube that is about halfway full, as a less full tube
makes the mixer vibrate excessively.

\begin{enumerate}
\item Seal and weigh the test tube containing the PDMS. It is
  recommended to do this in a weigh boat to keep the balance clean.
\item Prepare another 50mL test tube and fill it with enough water to
  weigh approximately as much as the tube of PDMS.
\item Remove both beaters from the hand mixer and set one aside.
\item Securely Tape both test tubes on opposite sides of the other
  beater. It is recommended to tape both the top and bottom of the
  test tubes.
\item Insert the beater into the mixer, and turn the mixer on low.
\item If the assembly seems stable, increase the mixer speed to
  medium-high and spin for 2-3 minutes
\item Stop the mixer and carefully remove the PDMS tube. The tube
  should contain no bubbles.
\item Very carefully pour the PDMS into the mold. The best way to do
  this is to hold the tube very close to the surface and pour
  extremely slowly. Failure to do this will result in bubbles forming
  in the mold.
\end{enumerate}

\subsection{Curing the PDMS}

\begin{enumerate}
\item Place the mold on a piece of aluminum foil on top of a hot
  plate. Glass is not recommended as the PDMS sticks well to glass.
\item If the mold is made from plastic, turn the hot plate on low, and
  let the PDMS cure for 2-4 hours. Take care not to go very far over
  the glass transition temperature of the plastic, in our case PLA
  ($60^\circ$ C).
\item If the mold is made from metal or high temperature plastic
  (melting point \textgreater $150^\circ$C), turn the hot plate to
  high/$150^\circ$C and cure for 15 minutes
\end{enumerate}
    
\subsection{De-Molding}
\begin{enumerate}
\item Remove the mold from the hot plate. Do not let it cool.
\item Take a pair of needle nose pliers and wrap electrical tape
    around the jaws
\item Using the wrapped needle nose pliers, carefully grip the glass
  fibre and slowly pull it through the block of PDMS.
\item If the fibre breaks, use the other end or any remaining fibre to
  attempt to pull it out. If no more fibre is available to pull on, or
  the coil is pulled along with the fibre, throw out the PDMS and try
  again.
\item Let the mold cool.
\item Using a scalpel, carefully cut the PDMS away from the sides of the mold. Take care not to cut or damage the copper wire.
\item Carefully peel the PDMS from the mold, taking care not to tear the gel.
\end{enumerate}

\end{document}
