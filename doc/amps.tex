\documentclass{article}
\title{We need a different amplifier}
\author{Michael Nolan}
\usepackage{circuitikz}
\usepackage{tikz}
\begin{document}
\maketitle
\section{Problem}
The current uCurrent style amplifier will not provide good data from the
uFluidics coil because the uCurrent relies on the source having a very high
output impedance, whereas the uFluidics coil has a low output impedance.
Illustrated below, the high impedance source converts a $1nA$ current into
$1mV$ and feeds it to the op amp to be amplified. Because the sense resistor
has a significantly lower impedance than the impedance of the current source,
most of the current flows through the sense resistor and generates a
comparatively large voltage across it. \\
\begin{circuitikz}
    \draw (0,3) node[label=\textit{High impedance source}] {} ;
    \draw (-1.5, -1) rectangle (2.5, 3);
    \draw (0,0)  node[ground] (gnd) {}
    to[I, i=$1nA$] (0,2)
    to[short] (2,2)
    to[R, l=$10M\Omega$] (2,0)
    to[short] (0,0) ;
    \draw (2,2)
    to[short] (4,2)
    node[label=$\approx1mV$] {}
    to[R, l=$1M\Omega$] (4,0)
    to[short] (2,0);
    \draw (6,2.5) node[op amp] (opamp) {}
    (opamp.+) to[short] (2,2);
    \draw 
    (opamp.-) to[R] (3,3) node[ground] (gnd) {};
    \draw 
    (opamp.-) to[short, *-] ++(0,2) coordinate (leftR)
    to[R] (leftR -| opamp.out)
    to[short, -*] (opamp.out);
    
\end{circuitikz}
\\
In contrast, in the circuit below, the sense resistor has a much higher
impedance than that of the source, so the source puts almost it's full voltage
across the sense resistor. However, since the source voltage is small, the
voltage would require far more amplification to get to the level of the above
example. The source below could be transformed to a current source in parallel
with the $100\Omega$ resistor similar to the above circuit, but since the
source impedance is much lower than the sense resistor, only a very small
amount of current will flow through the sense resitor, generating only a very
small voltage across it.
\\
\begin{circuitikz}
    \draw (0,3) node[label=\textit{Low impedance source}] {} ;
    \draw (-1.5, -1) rectangle (2.5, 3);
    \draw (0,0)  node[ground] (gnd) {}
    to[V, v=$1\mu V$] (0,2)
    to[R, l=$100\Omega$] (2,2)
    to[short] (4,2)
    node[label=$\approx1\mu V$] {}
    to[R, l=$1M\Omega$] (4,0)
    to[short] (0,0) ;
    \draw (6,2.5) node[op amp] (opamp) {}
    (opamp.+) to[short] (2,2);
    \draw 
    (opamp.-) to[R] (3,3) node[ground] (gnd) {};
    \draw 
    (opamp.-) to[short, *-] ++(0,2) coordinate (leftR)
    to[R] (leftR -| opamp.out)
    to[short, -*] (opamp.out);
    
\end{circuitikz}
\\ \\
How do we know that the coil has a low impedance? Intuitively, it makes sense
because it's simply a coil of wire, and it makes sense that the impedance would
be small (infinitely small, in the case of perfectly conductive wire. This
agrees with faraday's law of induction, where $$ V = -N \frac{d\Phi}{dt} $$
meaning that the coil is a voltage source (low impedance). Finally, this is
confirmed by experiment, when I hooked up a prototype uCurrent to the straw
coil and saw very little signal (indicating that the signal was a low impedance
voltage source). 
\\
\section{Solution}
Since this is the case, how much amplification do we need? The ADC on the
microcontroller has 12 bits of resolution, or 4096 different values. The ADC
linearly assigns these values to voltages between 0 and 3.3V, so we want the
resting voltage to be around 1.5V, and a decent amount of swing when a cell
passes through, say 400mV pk/pk. And from the COMSOL simulation the maximum
voltage produced was around 600nV. So we need amplification of
$$\frac{400mV}{600nV} = 660,000 = 116dB$$ We'll likely need to cascade op amps
to keep the feedback resistors within a reasonable value. Additionally, 600nV
is low enough that it approaches the thermal noise from the op amp itself, so
the op amps will need to be selected for exceptionally low noise, especially
low frequency noise since the cells will be traveling fairly slowly.
\\
\section{Op Amp Selection}
The op amps used for this amplifier need to have the following characteristics
\begin{itemize}
    \item \textless300nV 0.1-10Hz noise
    \item Very Low 1/f noise
    \item \textless1uV input offset voltage
\end{itemize}
The 0.1-10Hz noise needs to be minimized because that is where the signal will
be. The 1/f noise is less important, because it is more relevant at higher
frequencies, but should still be taken into account. The input offset voltage
will add a DC offset to each amplifier stage, and needs to be taken into
account to make sure each amplifier stage stays in the middle of its
input/output voltage range (so no clipping). \\ 
Most of the op amps in this category are ultra high precision, chopper
stabilized op amps. Some op amps that will likely work:\\ \\
\begin{tabular}{|l|c|c|c|}
    \hline
    Part & 0.1-10Hz noise $nV$ & 1/f noise $nV/\sqrt{Hz}$ & offset voltage $\mu V$ \\\hline
    MAX44243 & 117 & 9.0 & 1-5 \\\hline
    MAX44252 & 123 & 5.9 & 3-6 \\ \hline
    LT1028   & 90  & 1.0 & 20-80 \\ \hline
    OPA827   & 250 & 4.0 & 75-150 \\ \hline
    LT6018   & 30  & 1.2 & 7-50 \\ \hline
    LT1037   & 130 & 4.5 & 20-60 \\ \hline
    AD797    & 50  & 2.5 & 10-40 \\ \hline
    AD4528   & 97  & 5.6 & 0.3-2.5 \\ \hline
\end{tabular}\\ \\

\section{Stages}
The first amplifier stage should be the AD8428B ultra low noise instrumentation amplifier, as it has a lower 0.1-10Hz noise (40nV) than almost all of the op amps, and does not require external feedback. Therefore I propose using the AD8428B for the initial preamp with a gain of 2000, and the AD4528 for the output stage with gain $660000/2000 = 330$ for a total gain of $660000$.

\end{document}
